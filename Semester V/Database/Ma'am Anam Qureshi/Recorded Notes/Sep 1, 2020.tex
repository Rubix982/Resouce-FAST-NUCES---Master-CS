\documentclass{article}

\begin{document}

    \tableofcontents
    
    \section{Basic Introduction To Databases}

    Data has a lot of importance, and most of the trends of data coming out of countries are the ones which are most economically well and self sustained.
    For data, a straightforward definition, is a collection of facts and figures.
    It can,

    \begin{enumerate}
        \item Collection of information about anything
        \item Unprocessed information
        \item Information stored in form of tables and schemas
    \end{enumerate}

    Let's say data is a list of temperatures, or it is a reading of temperatures. Say it is,
    37, 52, 25, 27, 13, 18, 16

    Some useful insights can be,
    \begin{enumerate}
        \item We can find the mean
        \item See how far it is from the average of the rest of Pakistan
        \item Is it going up, is it going down?
        \item how can we process it?
    \end{enumerate}

    There is some slight difference between information and data.
    
    \begin{itemize}
        \item When you get useful insights from the data, it's called information.
        \item If the data is unprocessed, it remains as data.
    \end{itemize}

    \subsection{Outline}

    \begin{enumerate}
        \item Basic definitions: Data, information, Database, Database Management system, Database Systems
        \item Database systems vs 
    \end{enumerate}

    \subsection{Definitions}

    \begin{itemize}
        \item \label{databases}{\textbf{Databases}} - Collection of related data
        \item \textbf{Related Data} - Information that can be treated colletively about something, that serves as one entity
    \end{itemize}

    \subsection{Databases}

    Let's think about a University Management System. 
    We think of different problems,

    % \begin{enumerate}
    %     \item 
    % \end{enumerate}

    Some things to think about are as follows, 

    \begin{itemize}
        \item Features, some entitites
        \item Information about students
        \item Information about faculty
        \item Information about courses
        \item Information about accounts
        \item Information about departments
        \item Information about societies
        \item Information about transport
    \end{itemize}

    We can use a kind of diagram called, ERD, which is a \emph{Entity Related Diagram}

    \subsection{Database System}

    When you use databases as a software, it's just called a \textit{Database Management System}.

    A \textit{DBMS} is used to create a database, and is used to perform operations on it. 

    DMBS can be either MS SQL Server, IBM DB2 etc.

    We have something called a \textit{server side}, and then there is a \textit{user side}.

    User can be a,
    \begin{itemize}
        \item Client
        \item Admin
    \end{itemize}

    Let's say you made a ecommerce website, and we inserted a lot of data into it about the products. And then we placed the data on a server that serves and takes care of that information.
    For that, for the server, we use hosting service.
    We usually have backup servers to give the illusion of 24/7 availability of service to the users. So even if one database server goes down, we can still use another database to access the information.
    The admin can make changes to the database server at anytime in runtime mode. He can remove, read, update, insert, and so on and so forth.

    For example, when the client buys something, there is a read and write operation. 
    When you only read \textit{data}, it's called a query. When you read and write both, it's called a \textit{transaction}.



\end{document}