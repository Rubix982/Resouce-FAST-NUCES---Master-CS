\documentclass{article}

\begin{document}

    \tableofcontents

    \section{What are Algorithms?}

    The word algorithms comes from the Muslim scientist in the region of Iran, Khwarizmi. His date of birth is not known.

    He published a book called \textit{'Al Kitab Al Mukhtatasar fi hisab al-jab wa'l-muqabalah'}

    What is the algorithmic definitiion? It is as the folloing,
    \begin{itemize}
        \item An algorithm is any well-defined computational procedure that takes some values, or set of values, as input and produces some value, or set of values, as input.
        \item An algorithm is thus a sequence of computational steps that transform the input into output
    \end{itemize}

    What is an algorithm - a formal definition,
    \begin{itemize}
        \item An algorithm is a sequence of unambigious instructions for solving a problem, i.e, for obtaining a required output for an legitimate input in a finite amount of time
    \end{itemize}

    In the above, some keywords are,
    \begin{itemize}
        \item \textbf{Unnambiguity} - Must be very well defined
        \item \textbf{Finite amount of time} - should not done is some estimated amount of time
    \end{itemize}

    Another definition is,
    \begin{itemize}
        \item Recipe, process, method, technique, procedure, routine ...
    \end{itemize}

    \section{Why is the study of algorithms worthwhile?}

    To work on the time, and to demonstrate that our solution terminates, and with the correct answer,

    \begin{itemize}
        \item Suppose computers were infinitely fast and computer memory was free
        \item Would you have any reason to study algorithms?
        \item The answer is "YES"
        \item to demonstrate that your solution method terminates and does so with the correct answer
    \end{itemize}

    We study algorithms, because often we have limited resources, and we wish to complete the termination of the program with the correct answer, so that the fewest number of resources are used
    Putting in into summary,

    \begin{itemize}
        \item Processing time is a bounded resources
        \item Computers may be fsat but they are not infinitely fas. And memory maybe inexpensive, but it is not free.
        \item Use these resources wisely
        \item Insertion and Merge Sort
    \end{itemize}

    Let's take an example

    \begin{itemize}
        \item Suppose that computer A executes 10 billion instructions per second and computer B executes only 10 million instructions per seconds
        \item Insertion sort in machine language for computer A by world's crafftiest programmer resulting in 2n**2 time.
        \item Suppose an average programmer implements merge sort, using a high-level language with an inefficient compiler, taking 50 * n * log(n) \textit{Time}
        \item To sort 10million numbers, computer A takes,
    \end{itemize}
        \frac{2-(10^7)^2 instructions}{10^10 instructions/second} = 20,000 seconds (more than 5.5 hours). 
    \begin{itemize}
        \item While B takes,
    \end{itemize}
        \frac{50 * 10^7 * log(10^7) instructions}{10^7 instructions/second} ~ 1163 seconds (less than 20 minutes)

    \section{Design Of Algorithms}

    \begin{enumerate}
        \item Understand the problem
        \item Decide on: computational means, exact vs approximate solving, algorithm design technique
        \item Design an algorithm
        \item Prove correctness. Go to step 2, 3
        \item Analyze the algorithm
        \item Code the algorithm
    \end{enumerate}

    \subsection{Algorithm Design Techniques / Strategies}
    \begin{enumerate}
        \item Brute Force
        \item Divide And Conquer
        \item Decrease And Conquer
        \item Transform And Conquer
        \item Space and Time
        \item Branch and bounded
        \item Backtracking
        \item Dynamic Programming
        \item Greedy Approach
    \end{enumerate}

    \subsection{Analysis Of Algorithms}
    \begin{

    \subsection{Important Problem Types}
    \begin{enumerate}
        \item Sorting
        \item Serarchoing
        \item Numerical
        \item Graphical
        \item Combinatorial problems
        \item String searching
    \end{enumerate}

    \subsection{}

\end{document}