\documentclass{article}

\author{Saif Ul Islam}
\date{Sep 7th, 2020}
\title{Algorithms, Week 2, Lec 1}

\begin{document}

    \maketitle

    \tableofcontents

    \section{General Analysis Strategy}

    \begin{enumerate}
        \item T(n): Maximum line taken by algorithm to solve any input of size n
        \item T(n): Measure of goodness, how good or bad inidicated by function
        \item The function will indicate how good is a algorithm
        \item Conservative Definition - \textit{Worst Case}.
    \end{enumerate}

    What is \textit{T(n)}? It indicates the maximum time it would take for a machine to run that algorithm - the worst case scenario?

    \begin{enumerate}
        \item Form of T(n) (independent from machine)
        \item "Linear", "Cubic", "Quadratic" etc
        \item Bounds of T(n), upper bound, lower bound
        \item Large n is important, as n becomes larger and larger which algorithm is better
    \end{enumerate}

    \section{Running Time Analysis}

    \begin{enumerate}
        \item The running time depends upon the input size, e.g., n
        \item Different inputs of the same size may rsult in different running time
        \item Criteria for measuring running time
        \item Worst-Case Time (Maximum running time over legal inpout of size \textit{n})
        \item Criteria Worst-case time
        \item Let I denote an input instance
        \item Let |I| denote its length
        \item Let T(I) denote the running time of algorihtm on input I
    \end{enumerate}

    Some measurements are as follows,

    \begin{enumerate}
        \item Average Case Time - the average running time over all inputs of size n
        \item Let P(I) denote the probability of seeing this input
        \item Average case time is the weighted sum of running times with weights being the probabilities
    \end{enumerate}

    \subsection{Example: 2-Dimension Maxima}

    The car selection problem can be modelled this way: For each car we assocaite (x, y) pair where,

    \begin{enumerate}
        \item x is the speed of the car
        \item y is the negation of the price
    \end{enumerate}

    \subsubsection{Algorithm}

    MAXIMA(int n, Point P[1 ... n])
        for i <- 1 to n
        do maximal <- true:
            for j <- 1 to n
            do
                if(i != j) and  (P[i].x <= P[j].x) and (P[i].y <= P[j].y)
                    then maximal <- false;
                    break;
                if maximal == true
                    continue;

    \subsubsection{Analysis}

    To do \textit{Analysis}, just count the steps, 

\end{document}